% Created 2019-03-01 sex 23:36
% Intended LaTeX compiler: pdflatex
\documentclass[a4paper, 12pt]{article}
\usepackage[utf8]{inputenc}
\usepackage[T1]{fontenc}
\usepackage{graphicx}
\usepackage{grffile}
\usepackage{longtable}
\usepackage{wrapfig}
\usepackage{rotating}
\usepackage[normalem]{ulem}
\usepackage{amsmath}
\usepackage{textcomp}
\usepackage{amssymb}
\usepackage{capt-of}
\usepackage{hyperref}
\usepackage{minted}
\usepackage{fancyhdr}
\usepackage{lipsum}
\usepackage{indentfirst}
\usepackage[portuguese, ]{babel}
\usepackage{libertine}
\usepackage{tkz-graph}
\usepackage[usenames,dvipsnames]{xcolor}
\usepackage[left=3cm,bottom=3cm,top=2cm,right=2cm]{geometry}
\newcommand{\code}{\texttt}
\pagestyle{empty}
\fancyfoot[R]{\thepage}
\author{Rafael Campos Nunes}
\date{\textit{<2019-03-01 sex>}}
\title{Dataset Pipeline Server}
\hypersetup{
 pdfauthor={Rafael Campos Nunes},
 pdftitle={Dataset Pipeline Server},
 pdfkeywords={},
 pdfsubject={},
 pdfcreator={Emacs 26.1 (Org mode 9.1.9)},
 pdflang={Pt-Br}}
\begin{document}

\maketitle
\tableofcontents


\section{Introdução}
\label{sec:orga211026}

Esta aplicação faz parte de uma interface conjunta que é o \emph{dataset pipeline
client}, embora esteja em outro repositório. O objetivo desta é prover um
serviço através do protocolo HTTP que permita o envio de arquivos e o
armazenamento deles em disco local, além de também ter persistência em um banco
de dados local (sqlite).

\section{O Servidor}
\label{sec:org1ef9d94}

O servidor HTTP foi criado utilizando o pacote \emph{net/http} que fornece vários
recursos que auxiliam a disponibilização de um. O servidor recebe uma requisição
no endereço especificado pelo cliente e de acordo com a requisição retorna ou as
páginas e recursos que estão localizadas no diretório \emph{html} ou o resultado da
utilização da API REST.

Criei um servidor de rotas utilizando \emph{vanilla} Go com o pacote disponível na
biblioteca padrão o que pode ter feito a performance do servidor decair e até
mesmo ter vuneralibilidades de segurança. A utilização do gorilla/mux veio a
mente somente dias depois em que eu já tinha a estrutura do servidor pronta e
mudá-la tomaria um tempo que eu não tinha disponível.

O acesso ao servidor pode ser feita utilizando um browser ou a API REST,
disponibilizei duas formas para aumentar a usabilidade da ferramenta.

\section{O Banco de Dados}
\label{sec:org7874848}

Ao realizar a tarefa responsável por persistir os dados recebidos pelo servidor
foi escolhido o tipo de persistência que seria desenvolvida. Uma de duas opções
poderia ser adotada:

\begin{enumerate}
\item Persistência em SGBD
\item Persistência em memória
\end{enumerate}

A primeira opção foi escolhida e definida que seria um banco de dados relacional
fácil de instalar, configurar e utilizar. O SQLite foi escolhido porque se
adequa as três características descritas. Pode-se debater também que a
utilização de um banco de dados não relacional fosse melhor adequada para compor
a solução do problema haja visto que há a necessidade de inserir e acessar uma
grande quantidade de dados sem necessariamente ter relação entre os dados do
sistema. É um debate onde visualizo a vantagem na utilização do, a titulo de
exemplo, MongoDB.

Entretanto não o utilizei pelo fato de já ter montado a estrutura inicial, assim
como no servidor, e tinha pouco tempo hábil para mudá-la depois de visualizar a
melhor escolha. Além de também não ter em mente esta solução ao início do
desenvolvimento, onde se faz claro ser a melhor.

O código que remete ao banco de dados está presente no arquivo \emph{database.go} e
contém várias funções de auxílio às operações deste. Inicialização, inserção e
seleção facilitam o trabalho com o banco de dados pois abstraem detalhes. Além
dessas funções há também uma função específica para capturar o objeto \(sql.DB\),
com a assinatura \(GetHandler(params ...string)\), que permite realizar as funções
de inserção e seleção no banco de dados.

\section{A API}
\label{sec:org7b9bc72}

A API REST contém dois recursos, o recurso \emph{get} e o recurso \emph{new}. Os dois
recursos podem ser requisitados com um GET ao endereço do servidor utilizando
a seguinte sintaxe:

\begin{minted}[frame=lines,linenos=true]{html}
/v1/get?pk=<XXX>
/v1/new?filename=<XXX>&pk=<XXX>&score=<XXX>
\end{minted}

O servidor é encarregado ou de buscar o valor correspondendo a chave \emph{pk} ou
a salvar a nova entrada dada pelos atributos da requisição.

\section{Conclusão}
\label{sec:org99d8474}

Muitas coisas poderiam ser feitas, inclusiva a melhoria em performance utilizando
\emph{goroutines}. Entretanto, devido ao tempo hábil pequeno e eu estar me locomovendo
em viagens de ônibus (o que me tomou um dia e algumas horas) eu não consegui
realizar com o esmero que tinha imaginado.

A utilização do \(gorilla/mux\), como citado na seção do servidor, deve ser feita
para minar os problemas evidenciados e, além disso, simplificar o código de
upload de arquivos para persistir com maior rapidez e sem erros do servidor
atestando que muitos arquivos estão abertos - erro que ainda não consegui
resolver -
\end{document}
